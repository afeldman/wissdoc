\documentclass{wissdoc}

\usepackage[utf8]{inputenc}   % Unicode input

% Index-Datei öffnen
\makeindex

% Shortcuts, nicht unbedingt notwendig
%\input{shortcut}

%PDF Settings
\definecolor{listingsbg}{rgb}{0.9,0.9,0.9} 
\definecolor{darkblue}{rgb}{0,0,.5}

\hypersetup{
      pdftoolbar=true,        % show Acrobat’s toolbar?
      pdfmenubar=true,        % show Acrobat’s menu?
      pdffitwindow=true,     % window fit to page when opened
      pdfstartview={Fit},    % fits the width of the page to the window
      pdftitle={Robust and Accurate Self Calibration of Stereo Camera Systems},    % title
      pdfauthor={Anton Feldmann},     % author
      pdfsubject={Accurate Calibration},   % subject of the document
      pdfcreator={LiveTex},   % creator of the document
      pdfproducer={Anton Feldmann}, % producer of the document
      pdfkeywords={epipolar geometrie, bundel adjustement, prediction bands}, % list of keywords
      pdfnewwindow=true,      % links in new window
      pdfpagelabels=true,% view TeX pagenumber in PDF reader
      pdftex=true,
      pdfborder={0 0 0.5},
      colorlinks=true,       % false: boxed links; true: colored links
      linkcolor=darkblue,          % color of internal links
      citecolor=blue,        % color of links to bibliography
      filecolor=magenta,      % color of file links
      urlcolor=darkblue,
      bookmarks=true,         % show bookmarks bar?
      unicode=true,          % non-Latin characters in Acrobat’s bookmarks 
      breaklinks=true,  
      menucolor=darkblue,
      pagebackref=true,
      plainpages=false,% correct hyperlinks
      bookmarksopen=true,    
      bookmarksnumbered=true,
      raiselinks=true,%
      %bookmarksopenlevel=1,%
      hyperindex=true}


% citationformat
\bibpunct{[}{]}{;}{a}{,}{}


\author{ }%%insert authorname
\date{\today}%% insert date std: today

% open glossary
\newglossary[slg]{symbolslist}{syi}{syg}{Symbol List}

\renewcommand*{\glspostdescription}{}%Den Punkt am Ende jeder Beschreibung deaktivieren
\makeglossaries

\input{glossary}

\setlength{\headheight}{1.1\baselineskip}

%%%%%%%%%%%%%%%%%%%%%%%%%%%%%%%%%%%%%%%%%%%%%%%%%%%%%%%%%%%%%%%%%%%%%%%%%%%%%%%%%%%%%%%%%
%                                     Start Document
%%%%%%%%%%%%%%%%%%%%%%%%%%%%%%%%%%%%%%%%%%%%%%%%%%%%%%%%%%%%%%%%%%%%%%%%%%%%%%%%%%%%%%%%%

\begin{document}

\onehalfspace

\graphicspath{{img/}}%Include graphic path default is .ProjectRoot/img 

\lstset{ %\cleardoublepage
language=Matlab,                % choose the language of the code
basicstyle=\ttfamily\small,     % the size of the fonts that are used for
				% the code
numbers=left,                   % where to put the line-numbers
numberstyle=\footnotesize\ttfamily,      % the size of the fonts that are used
				%for the line-numbers
numbersep=5pt,
stepnumber=2,                   % the step between two line-numbers. If it's 1
				% each line will be numbered
numbersep=5pt,                  % how far the line-numbers are from the code
xleftmargin=1cm,xrightmargin=1cm,
linewidth=15cm,
backgroundcolor=\color{white},  % choose the background color. You must add
showspaces=false,               % show spaces adding particular underscores
showstringspaces=false,         % underline spaces within strings
showtabs=false,                 % show tabs within strings adding particular
				% underscores
frame=single,	                % adds a frame around the code
tabsize=2,	                % sets default tabsize to 2 spaces
captionpos=b,                   % sets the caption-position to bottom
breaklines=true,                % sets automatic line breaking
breakatwhitespace=false,        % sets if automatic breaks should only happen at
				% whitespace
title=\lstname,                 % show the filename of files included with
resetmargins=true,
breaklines=true,
commentstyle=\color{blue},
keywordstyle=\color{blue}\ttfamily,
escapeinside={\%*}{*)},         % if you want to add a comment within your code
}

\frontmatter
\pagenumbering{roman}
\newsavebox{\Prof}
\savebox{\Prof}{ == insert professor name == }

\begin{titlepage}
\begin{center}
\underline{\hspace{15cm}} 
\vskip 3cm
{\huge\bfseries \par{ == insert title == } }\\
\vskip 3cm
\underline{\hspace{15cm}} 

\vskip 5cm

\textbf{ == insert name == }\\
== date of handin ==

\vskip 7cm

== kind of document e.g. Dissatertion ==\\

\end{center}
\vfill
\end{titlepage}
%% Titelseite Ende
%change Titlepage

%%% insert here the pages befor the table of contend. e.g. thanks

\blankpage % Leerseite auf Dankesagungsrückseite


%
%% *************** Hier geht's los ****************
%% ++++++++++++++++++++++++++++++++++++++++++
%% Verzeichnisse
%% ++++++++++++++++++++++++++++++++++++++++++
\parskip 0pt\tableofcontents % toc bitte einzeilig
\blankpage
%% ++++++++++++++++++++++++++++++++++++++++++
%% Abbildungs und tabellenverzeichnis
%% ++++++++++++++++++++++++++++++++++++++++++
\listoffigures
\blankpage
\listoftables
\blankpage
\listofalgorithms
\blankpage

%% ++++++++++++++++++++++++++++++++++++++++++
%% Glossary
%% ++++++++++++++++++++++++++++++++++++++++++
%Glossar ausgeben
\printglossary[style=altlist,title=Glossary]
%Abkürzungen ausgeben
\printglossary[type=\acronymtype,style=index,title=List of Abbreviations,toctitle=List of Abbreviations]
%Symbole ausgeben
\printglossary[type=symbolslist,style=long,title=List of Symbols,toctitle=List of Symbols]
\blankpage

% Abstract
\include{ == abstract document identifier == }
\blankpage


\mainmatter

%%insert here the document identifier 
\include{abstract}
\include{chapter1}

%% ++++++++++++++++++++++++++++++++++++++++++
%% Anhang
%% ++++++++++++++++++++++++++++++++++++++++++

\appendix
\include{apandix1}

%% ++++++++++++++++++++++++++++++++++++++++++
%% Literatur
%% ++++++++++++++++++++++++++++++++++++++++++
%  mit dem Befehl \nocite werden auch nicht 
%  zitierte Referenzen abgedruckt
%\cleardoublepage
\blankpage
\addcontentsline{toc}{chapter}{\bibname}
\nocite{*} % nur angeben, wenn auch nicht im Text zitierte Quellen 
           % erscheinen sollen
\bibliographystyle{apalike-doi}
\bibliography{ == Bibfile identifier == }

%% ++++++++++++++++++++++++++++++++++++++++++
%% Index
%% ++++++++++++++++++++++++++++++++++++++++++
%\cleardoublepage
\blankpage
\printindex            % Index, Stichwortverzeichnis

\blankpage

\backmatter
\onecolumn

% Die folgende Erklärung ist für Diplomarbeiten Pflicht
% (siehe Prüfungsordnung), für Studienarbeiten nicht notwendig
\include{ garantee }
\blankpage % Leerseite auf Lebenslauf

\end{document}
